\section{ShinyApp}\label{Sec:shiny}

\texttt{shiny} is an \texttt{R} package that allows the user to write interactive web applications. These are especially helpful when delivering information to people with no coding experience in a very user-friedly way.

As described in the official site of \texttt{shiny}, in its most basic version an App is contained in a single script called \textit{app.R}. As the Apps become more and more complicated one can write the code in two scripts, \textit{ui.R} and \textit{server.R}, or even further split these into thematic scripts.
\\
The App structure is divided into two main elements:
\begin{itemize}
    \item \textit{ui}: The user interface object determines the appearance of the App itself. Here the possible inputs and the ouputs will be defined.
    \item \textit{server}: The server function uses the input values chosen in the App to produce the outputs indicated in the user interface. 
\end{itemize}

Finally the App will be called using the function \texttt{shinyApp(ui, server)}.

Because of its interactiveness and flexibility a ShinyApp was built for this project in order to enable the final user a chance to a first hand analysis of the data. The code for the App can be found in the Appendix at subsection \ref{subsec:codeshiny}. It is divided into four tabs: the first one is just introductory, presenting the project and the App itself; the second display Berlin maps and descriptive statistics according to section \ref{Sec:Exploratory}; the third relates to price analysis explained in section \ref{Sec:Price analysis}; the fourth and final one enables the user to perform clustering, like illustrated in section \ref{Sec:cluster}. The ShinyApp is reachable at this link: \url{https://silviav.shinyapps.io/airbnb-berlin/}.

