\section{Exploratory Data Analysis}\label{Sec:Exploratory}

The exploratory data analysis will be divided in two parts: subsection \ref{subsec:descriptive} will show the descriptive statistics calculated either numeric or categorical variables; subsection \ref{subsec:distrplots} will then display the distribution plots. 
Correlation will also be calculated, but only with respect to price. This subject will be dealt with in subsection \ref{subsec:corr}.

\subsection{Descriptive statistics}\label{subsec:descriptive}

Descriptive statistics make the interpretation of the data easier by giving grouping it and thus providing a shorter representation of it \citep{descrstat:2014}.

As in \cite{descrstat:2014} explained there are three general types of descriptive statistics:
\begin{enumerate}
\item Measures of central tendency
\item Measures of spread
\item Graphical displays
\end{enumerate}

This subsection will focus on the first two, while subsection 
\ref{subsec:distrplots} will deal with the third.

Most of these statistics are usually applied to continuous data, sometimes even to numerical discrete data, but not to categorical variables. Therefore the function to calculate descriptive statistics has been split in two.

The first part calculate both measures of central tendency and spread of all numerical variables thanks to the function \texttt{apply}. The function \texttt{apply(X, MARGIN, FUN, ...)} calculates the function in \texttt{FUN} for all rows (\texttt{MARGIN = 1}) or columns (\texttt{MARGIN = 2}) for the data in \texttt{X}. One can add additional arguments, like the quantile probabilites in our case.


\lstinputlisting[language=R, firstline=1, lastline=17, firstnumber=1, escapechar=|, caption={|\textbf{\href{https://github.com/silvia-ventoruzzo/SPL-WISE-2018/blob/master/Helpers/descriptive_statistics.R}{descriptive\_statistics.R}}|}]{../Helpers/descriptive_statistics.R}

Part of the results can be seen in table \ref{table:descrstatnum}.



\begin{table}[H]
\centering
\centering
\begin{tabular}{lrrrrrrrr}
  \hline \hline
variable & min & 1Q & median & 3Q & max & iqr & mean & sd \\ 
  \hline
price & 0.00 & 30.00 & 47.00 & 70.00 & 9000.00 & 40.00 & 67.69 & 210.53 \\ 
  review\_scores\_rating & 0.00 & 84.00 & 95.00 & 100.00 & 100.00 & 16.00 & 77.26 & 37.09 \\ 
   \hline \hline
\end{tabular}
\caption{Sample of descriptive table for numeric variables \protect\includegraphics[scale=0.05]{qletlogo.pdf} {\href{https://github.com/silvia-ventoruzzo/SPL-WISE-2018/blob/master/Helpers/descriptive_statistics.R}{descriptive\_statistics.R}}}
\label{table:descrstatnum}
\end{table}

For categorical variables the above used statistics do not work, therefore frequencies and proportions of each factor were calculated. The mode is then simply the factor with the highest frequency.

\lstinputlisting[language=R, firstline=20, lastline=50, firstnumber=19, caption={|\textbf{\href{https://github.com/silvia-ventoruzzo/SPL-WISE-2018/blob/master/Helpers/descriptive_statistics.R}{descriptive\_statistics.R}}|}]{../Helpers/descriptive_statistics.R}


\subsection{Distribution plots}\label{subsec:distrplots}

Also for the distribution plots we distinguish between numerical and categorical variables. In the former case a density plot has been chose, while in the latter a bar plot.

Unfortunately, many numeric variables present outliers, which, in the case of very skewed data, have been excluded for better visualization. An example can be seen in figure \ref{figure:numdistr}.

\begin{figure}[H]
\centering
\subfloat[Complete]{\includegraphics[height=7cm]{price_distribution_complete.pdf}}
\subfloat[Without outliers]{\includegraphics[height=7cm]{price_distribution_nooutliers.pdf}}
\caption{Distribution of the variable price \protect\includegraphics[scale=0.05]{qletlogo.pdf} {\href{https://github.com/silvia-ventoruzzo/SPL-WISE-2018/blob/master/Helpers/distribution_plot.R}{distribution\_plot.R}}}
\centering
\label{figure:numdistr}
\end{figure}

For the categorical variables a bar plot is more appropriate, since variable can only assume a limited amount of values,and usually only a few, like in the case displayed in figure \ref{figure:catdistr}.

\begin{figure}[H]
\begin{center}
\includegraphics[width=0.5\textwidth, keepaspectratio]{room_type_distribution.pdf} \\
\caption{Distribution of the variable station\_count \protect\includegraphics[scale=0.05]{qletlogo.pdf} {\href{https://github.com/silvia-ventoruzzo/SPL-WISE-2018/blob/master/Helpers/distribution_plot.R}{distribution\_plot.R}}}
\label{figure:catdistr}
\end{center}
\end{figure}

Having succesfully created spatial polygons for Berlin, we can also map the distribution of the variables across the city. For example, in figure \ref{figure:priceavg} one can see how the average of price in each neighbourhood is distributed across neighbourhoods.

\begin{figure}[H]
\begin{center}
\includegraphics[width=0.5\textwidth, keepaspectratio]{price_map_distr} \\
\caption{Distribution of the average of price across Berlin's districts}
\label{figure:priceavg}
\end{center}
\end{figure}